%!TEX TS-program = xelatex
%!TEX encoding = UTF-8 Unicode
\documentclass[12pt,a4paper]{article}
\usepackage{geometry} % 設定邊界
\geometry{
  top=1in,
  inner=1in,
  outer=1in,
  bottom=1in,
  headheight=3ex,
  headsep=2ex
}
\usepackage{fontspec} % 允許設定字體
\usepackage{xeCJK} % 引入可以設置中文字型的套件
\setCJKmainfont{LiHei Pro} % 設定中文字型
\setmainfont{Georgia} % 設定英文字型
\setromanfont{Georgia} % 字型
\setmonofont{Courier New}
\linespread{1.2}\selectfont % 行距
\XeTeXlinebreaklocale "zh" % 針對中文自動換行
\XeTeXlinebreakskip = 0pt plus 1pt % 字與字之間加入0pt至1pt的間距,確保左右對整齊
\parindent 0em % 段落縮進
\setlength{\parskip}{20pt} % 段落之間的距離

\title{\huge 資訊安全 Final Project} % 設置標題,使用巨大字體
\author{吳嘉瑋、江青霞、廖健智} % 設置作者
\date{2017/05/19} % 設置日期,預設值為系統當天的日期。
\usepackage{titling}
\setlength{\droptitle}{-8em} % 將標題移動至頁面的上面
\usepackage{listings}
\usepackage{multicol}
\usepackage{color}
\usepackage{wrapfig} %引入可以在多行架構的文章中插入圖片的套件

\begin{document}
\clearpage
\maketitle % 顯示標題,下達這個指令才會把標題印出來。

% 設定多欄式頁面
\begin{multicols}{2}

\section{Introduction}
Many online services let users query large public 
datasets: some examples include restaurant sites, 
product catalogs, stock quotes, and searching for 
directions on maps. In these services, any user can 
query the data, and the datasets themselves are not 
sensitive. However, web services can infer a great 
deal of identifiable and sensitive user information 
from these queries, such as her current location, 
political affiliation, sexual orientation, income, 
etc.
		
\section{Proposed Method}
\subsection{Splinter}
\subsubsection{Architecture}
There are two main principals in Splinter: the user 
and the providers. Each provider hosts a copy of 
the data. Providers can retrieve this data from a 
public repository or mirror site. For a given user 
query, all the providers have to run it on the same 
view of the data. 
  A user splits her query into shares, using the 
Splinter client, and submits each share to a 
different provider. The user can select any 
providers of her choice that host the dataset. The 
providers use their shares to execute the user’s 
query over the cleartext public data, using the 
Splinter provider library. As long as one provider 
is honest (does not collude with others), the 
user’s sensitive information in the original query 
remains private. When the user receives the 
responses from the providers, she combines them to 
obtain the final answer to her original query.

\subsubsection{Security Goals}
The goal of Splinter is to hide sensitive
parameters in a user’s query.Specifically, Splinter 
lets users run parametrized queries, where both the 
parameters and query results are hidden from 
providers.

Splinter supports a subset of the SQL language, 
hides the information represented by the questions 
marks and the query’s results, but the column names 
being selected and filtered are not hidden.
\subsubsection{Threat Model}

\section{Case Studies}
\section{Related Work}
\section{Experimental Result}
\section{Conclusion}

\section{Reference}
[1] E. Boyle, N. Gilboa, and Y. Ishai. Function 
secret sharing. In Proceedings of the 34th Annual 
International Confer- ence on the Theory and 
Applications of Cryptographic Techniques 
(EUROCRYPT), pages 337–367. Sofia, Bul- garia, Apr. 
2015.

[2] N. Gilboa and Y. Ishai. Distributed point 
functions and their applications. In Proceedings of 
the 33rd Annual International Conference on the 
Theory and Applications of Cryptographic Techniques 
(EUROCRYPT), pages 640– 658. Copenhagen, Denmark, 
May 2014.



\end{multicols}







\end{document}
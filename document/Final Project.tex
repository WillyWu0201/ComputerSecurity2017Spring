%!TEX TS-program = xelatex
%!TEX encoding = UTF-8 Unicode
\documentclass[12pt,a4paper]{article}
\usepackage{geometry} % 設定邊界
\geometry{
  top=1in,
  inner=1in,
  outer=1in,
  bottom=1in,
  headheight=3ex,
  headsep=2ex
}
\usepackage{fontspec} % 允許設定字體
\usepackage{xeCJK} % 引入可以設置中文字型的套件
\setCJKmainfont{LiHei Pro} % 設定中文字型
\setmainfont{Georgia} % 設定英文字型
\setromanfont{Georgia} % 字型
\setmonofont{Courier New}
\linespread{1.2}\selectfont % 行距
\XeTeXlinebreaklocale "zh" % 針對中文自動換行
\XeTeXlinebreakskip = 0pt plus 1pt % 字與字之間加入0pt至1pt的間距,確保左右對整齊
\parindent 0em % 段落縮進
\setlength{\parskip}{20pt} % 段落之間的距離

\title{\huge 資訊安全 Final Project} % 設置標題,使用巨大字體
\author{5105056013 吳嘉偉、5105056029 江青霞、5105056019 廖健智} % 設置作者
\date{2017/06/25} % 設置日期,預設值為系統當天的日期。
\usepackage{titling}
\setlength{\droptitle}{-8em} % 將標題移動至頁面的上面
\usepackage{listings}
\usepackage{multicol}
\usepackage{color}
\usepackage{wrapfig} %引入可以在多行架構的文章中插入圖片的套件

\begin{document}
\clearpage
\maketitle % 顯示標題,下達這個指令才會把標題印出來。

% 設定多欄式頁面
\begin{multicols}{2}

\section{Introduction}
Many online services let users query large public 
datasets: some examples include restaurant sites, 
product catalogs, stock quotes, and searching for 
directions on maps. In these services, any user can 
query the data, and the datasets themselves are not 
sensitive. However, web services can infer a great 
deal of identifiable and sensitive user information 
from these queries, such as her current location, 
political affiliation, sexual orientation, income, 
etc.
		
\section{Proposed Method}
\subsection{Splinter}
\subsubsection{Architecture}
There are two main principals in Splinter: the user 
and the providers. Each provider hosts a copy of 
the data. Providers can retrieve this data from a 
public repository or mirror site. For a given user 
query, all the providers have to run it on the same 
view of the data. 
  A user splits her query into shares, using the 
Splinter client, and submits each share to a 
different provider. The user can select any 
providers of her choice that host the dataset. The 
providers use their shares to execute the user’s 
query over the cleartext public data, using the 
Splinter provider library. As long as one provider 
is honest (does not collude with others), the 
user’s sensitive information in the original query 
remains private. When the user receives the 
responses from the providers, she combines them to 
obtain the final answer to her original query.

\subsubsection{Security Goals}
The goal of Splinter is to hide sensitive 
parameters in a user's query. Specifically, 
Splinter lets users run parametrized queries, where 
both the parameters and query results are hidden 
from providers.

Splinter supports a subset of the SQL language, 
hides the information represented by the questions 
marks and the query’s results, but the column names 
being selected and filtered are not hidden.
\subsubsection{Threat Model}
Splinter keeps the parameters in the user’s query 
hidden as long as at least one of the user-chosen 
providers does not collude with others. Splinter 
also assumes these providers are honest but 
curious: a provider can observe the inter- actions 
between itself and the client, but Splinter does 
not protect against providers returning incorrect 
results or maliciously modifying the dataset.

It assume that the user communicates with each 
provider through a secure channel (e.g., using 
SSL), and that the user’s Splinter client is 
uncompromised. The cryptographic assumptions are 
standard. They only assume the existence of one-way 
functions in our two-provider implementation. In 
our implementation for multiple providers, the 
security of Paillier encryption is also 
assumed.

\section{Case Studies}
\section{Related Work}
\subsection{PIR(Private Information Retrieval ) 
systems}
Splinter is most closely related to systems that 
use Private Information Retrieval(PIR) to query a 
database privately. In PIR, a user queries for the 
ith record in the database, and the database does 
not learn the queried index i or the result. Much 
work has been done on improving PIR protocols. Work 
has also been done to extend PIR to return multiple 
records, but it is computationally expensive. Our 
work is most closely related to the system in, 
which implements a parametrized SQL-like query 
model similar to Splinter using PIR. However, 
because this system uses PIR, it has up to 10⇥ more 
round trips and much higher response times for 
similar queries.

Popcorn[4] is a media delivery service that uses 
PIR to hide user consumption habits from the 
provider and content distributor. However, Popcorn 
is optimized for streaming media databases, like 
Netflix, which have a small number (about 8000) of 
large records.
The systems above have a weaker security model: all
the providers need to be honest. Splinter only 
requires one honest provider, and it is more 
practical because it extends Function Secret 
Sharing (FSS) [5, 6], which lets it execute 
complex operations such as sums in one round trip 
instead of only extracting one data record at a 
time.
\subsection{Garbled circuits}
Systems such as Embark [7], BlindBox [8], and 
private shortest path computation systems [9] 
use garbled circuits [10, 11] to perform private 
computation on a single untrusted server. Even with 
improvements in practicality [12], these techniques 
still have high computation and bandwidth costs for 
queries on large datasets because a new garbled 
circuit has to be generated for each query. 
(Reusable garbled circuits [13] are not yet 
practical.) For example, the recent map routing 
system by Wu et al. [9] uses garbled circuits and 
has 100⇥ higher response time and 10⇥ higher
bandwidth cost than Splinter.
\subsection{Encrypted data systems}
Systems that compute on encrypted data, such as CryptDB [14], Mylar [15], SPORC [16], Depot [17], and SUNDR [18], all try to protect private data against a server compromise, which is a different problem than what Splinter tries to solve. CryptDB is most similar to Splinter because it allows for SQL-like queries over encrypted data. However, all these systems protect against a single, potentially compromised server where the user is storing data privately, but they do not hide data access patterns. In contrast, Splinter hides data access patterns and a user’s query parameters but is only designed to operate on a public dataset that is hosted at multiple providers.
\subsection{ORAM(Oblivious RAM) systems}
Splinter is also related to systems that use Oblivious RAM [19, 20]. ORAM allows a user to read and write data on an untrusted server without revealing her data access patterns to the server. However, ORAM cannot be easily applied into the Splinter setting. One main requirement of ORAM is that the user can only read data that she has written. In Splinter, the provider hosts a public dataset, not created by any specific user, and many users need to access the same dataset.
\section{Experimental Result}
\section{Conclusion}
Splinter is a new private query system that protects sensitive parameters in SQL-like queries while scaling to realistic applications. Splinter uses and extends a recent cryptography primitive, Function Secret Sharing (FSS), allowing it to achieve up to an order of magnitude better performance compared to previous private query systems. [1] Develop protocols to execute complex queries with low computation and bandwidth. As a proof of concept, it has evaluated Splinter with three sample applications—a Yelp clone, map routing, and flight search—and showed that Splinter has low response times from 50 ms to 1.6 seconds with low hosting costs.

\section{Reference}
[1] Splinter: Practical Private Queries on Public 
Data Frank Wang, Catherine Yun, Shafi Goldwasser, 
Vinod Vaikuntanathan, Matei Zaharia† MIT CSAIL, 
†Stanford InfoLab

[2] E. Boyle, N. Gilboa, and Y. Ishai. Function 
secret sharing. In Proceedings of the 34th Annual 
International Confer- ence on the Theory and 
Applications of Cryptographic Techniques 
(EUROCRYPT), pages 337–367. Sofia, Bul- garia, Apr. 
2015.

[3] N. Gilboa and Y. Ishai. Distributed point 
functions and their applications. In Proceedings of 
the 33rd Annual International Conference on the 
Theory and Applications of Cryptographic Techniques 
(EUROCRYPT), pages 640– 658. Copenhagen, Denmark, 
May 2014.

[4] T.Gupta, N.Crooks, S.T.Setty, L.Alvisi, and  M.Walfish. Scalable and private media consumption with Popcorn. In Proceedings of the 13th Symposium on Networked Systems Design and Implementation (NSDI), pages 91–107, Santa Clara, CA, Mar. 2016.

[5] E. Boyle, N. Gilboa, and Y. Ishai. Function secret sharing. In Proceedings of the 34th Annual International Confer- ence on the Theory and Applications of Cryptographic Techniques (EUROCRYPT), pages 337–367. Sofia, Bul- garia, Apr. 2015.

[6] N. Gilboa and Y. Ishai. Distributed point functions and their applications. In Proceedings of the 33rd Annual International Conference on the Theory and Applications of Cryptographic Techniques (EUROCRYPT), pages 640– 658. Copenhagen, Denmark, May 2014.

[7] C. Lan, J. Sherry, R. A. Popa, S. Ratnasamy, and Z. Liu. Embark: Securely outsourcing middleboxes to the cloud. In Proceedings of the 13th Symposium on Networked Sys- tems Design and Implementation (NSDI), pages 255–273, Santa Clara, CA, Mar. 2016.

[8] J. Sherry, C. Lan, R. A. Popa, and S. Ratnasamy. Blind- box: Deep packet inspection over encrypted traffic. In Proceedings of the 2015 ACM SIGCOMM, pages 213–226, London, United Kingdom, Aug. 2015.

[9] D. J. Wu, J. Zimmerman, J. Planul, and J. C. Mitchell. Privacy-preserving shortest path computation. In Proceed- ings of the 2016 Annual Network and Distributed System Security Symposium, San Diego, CA, Feb. 2016.

[10] M. Bellare, V. T. Hoang, and P. Rogaway. Foundations of garbled circuits. In Proceedings of the 19th ACM Confer- ence on Computer and Communications Security (CCS), pages 784–796, Raleigh, NC, Oct. 2012.

[11] S.Goldwasser. Multipartycomputations: pastandpresent. In Proceedings of the 16th Annual ACM Symposium on Principles of Distributed Computing (PDC), pages 1–6, 1997.

[12] M.Bellare, V.T.Hoang, S.Keelveedhi, andP.Rogaway. Efficient garbling from a fixed-key blockcipher. In Proceedings of the 34th IEEE Symposium on Security and Privacy, pages 478–492, San Francisco, CA, May 2013.

[13] S. Goldwasser, Y. Kalai, R. A. Popa, V. Vaikuntanathan, and N. Zeldovich. Reusable garbled circuits and succinct functional encryption. In Proceedings of the 45th Annual ACM Symposium on Theory of Computing (STOC), pages 555–564, Palo Alto, CA, June 2013.

[14] R.A.Popa, C.M.S.Redfield, N.Zeldovich, andH.Balakrishnan. CryptDB: Protecting confidentiality with encrypted query processing. In Proceedings of the 23rd ACM Symposium on Operating Systems Principles (SOSP), pages 85–100, Cascais, Portugal, Oct. 2011.

[15] R. A. Popa, E. Stark, J. Helfer, S. Valdez, N. Zeldovich, M. F. Kaashoek, and H. Balakrishnan. Building web applications on top of encrypted data using Mylar. In Proceedings of the 11th Symposium on Networked Systems Design and Implementation (NSDI), pages 157–172, Seattle, WA, Apr. 2014.

[16]A.J.Feldman, W.P.Zeller, M.J.Freedman, andE.W.Felten. SPORC: Group collaboration using untrusted cloud resources. In Proceedings of the 9th Symposium on Operating Systems Design and Implementation (OSDI), Vancouver, Canada, Oct. 2010.

[17] P. Mahajan, S. Setty, S. Lee, A. Clement, L. Alvisi, M. Dahlin, and M. Walfish. Depot: Cloud storage with minimal trust. In Proceedings of the 9th Symposium on Operating Systems Design and Implementation (OSDI), Vancouver, Canada, Oct. 2010.

[18] J. Li, M. Krohn, D. Mazières, and D. Shasha. Secure untrusted data repository (SUNDR). In Proceedings of the 6th Symposium on Operating Systems Design and Implementation (OSDI), pages 91–106, San Francisco, CA, Dec. 2004.

[19] J. R. Lorch, B. Parno, J. Mickens, M. Raykova, and J. Schiffman. Shroud: Ensuring private access to largescale data in the data center. In Proceedings of the 11th USENIX Conference on File and Storage Technologies (FAST), pages 199–213, San Jose, CA, Feb. 2013.

[20] E. Stefanov, M. van Dijk, E. Shi, C. Fletcher, L. Ren, X. Yu, and S. Devadas. Path ORAM: An extremely simple oblivious RAM protocol. In Proceedings of the 20th ACM Conference on Computer and Communications Security (CCS), Berlin, Germany, Nov. 2013.
\end{multicols}
\end{document}